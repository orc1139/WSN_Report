\chapter{Introduction}
\label{introduction}
\noindent An important part of city planning involves estimating the amount of traffic a certain street or city district will support.
Cars going to a certain destination will ultimately have to find a parking space. 
This means that obtaining information on parking space usage could greatly contribute to improvement of urban spaces and regulate parking pricing.

For instance, a study conducted between 1927 and 2001 \cite{cruising_for_parking_2007} in several cities around the world showed 
that between 8 and 75\% of the traffic on the streets in the study areas were looking for parking.
The average time spent cruising for a parking space in urban public parking areas varies from 3.5 to 12 minutes, 
with distance traveled ranging from 1 km to over 1.5 km. 
When multiplied by the number of cars looking for a parking space, this translates into a considerable amount traffic, wasted fuel, and unnecessary air pollution.

A parking space monitoring system could help alleviate such problems. However, current vacant parking space systems are almost exclusively used in multi-story car parks.
The reason behind it is that the typical sensor-based parking space detection system are costly and require a considerable amount of time and effort for installation.
Intrusive sensors (e.g. pneumatic tubes, weight-in-motion sensors, piezoelectric cables) need to be embedded in floor and non-intrusive sensors are wired to 
the ceiling or walls (e.g. ultrasonic sensors, CCTV-based detection).

In this project we want to demonstrate a cost-effective alternative to such systems with a prototype using interconnected wireless sensors.
By using Crossbow MicaZ motes, the sensors could be used in parking lots without roof, or even on the streets for curb parking slots.
The statistics are then fed to a web server that could be accessed, for example, by the parking lot operator or by smart devices on cars.
This means that such a system could provide benefits for both parties:

\begin{table*}[ht]
\centering
{\normalsize
\hfill{}
  \begin{tabular}{| l | l |}
    \hline
    Benefits for operators & Benefits for drivers \\ \hline
    Near-real-time parking lot occupancy information. & Guided assistance in finding a free space.\\
    More efficient and safer traffic flow. & Reduced carbon dioxide emissions.\\
    Parking usage trend analysis. & Lower fuel consumption.\\
    Improved space utilization. & Cruising time savings.\\
    Valuable statistics. & Less frustration.\\ \hline
  \end{tabular}}
\hfill{}
\caption{Examples of benefits for a cost-effective Vacant Parking Space Detector system.}
\label{tb:project_benefits}
\end{table*} 