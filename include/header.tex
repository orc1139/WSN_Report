%\documentclass[a4paper,11pt,openany]{book}
\documentclass[a4paper,
	11pt,
	BCOR5mm,
	DIV12,
	oneside,
	chapterprefix,
	pagesize,
        numbers=noenddot,
        cleardoublepage=plain,
        openright,
        bibliography=totoc,
        bibliography=numbered,
        index=totoc,
        listof=totoc,
        pagesize=pdftex,
        listof=numbered]{scrreprt}


% NOTE: if you get an error such as "! Missing $ inserted" it means a character that can only
% be used in the mathematics was inserted in normal text. 
% If you did not intend to use mathematics mode, then perhaps you are trying to use a special 
% character that needs to be entered in a different way.
% This can also happen if you use the wrong character encoding.
% There are several character encoding formats, make sure to pick the right one:
% 		See http://en.wikibooks.org/wiki/LaTeX/Special_Characters#Input_encoding
% Here are some examples. Please use only ONE:
%\usepackage[latin1]{inputenc}	% Old Unix environment.
%\usepackage[ansinew]{inputenc}	% Windows environment.
\usepackage[utf8]{inputenc}	% Modern Unix environment.

\usepackage{enumitem}	% For the [noitemsep] in enums and itemizes
\usepackage{sidecap}	% Needed to place a caption beside a figure or table.
\usepackage{tabu}		% Needed to allow tables.

\usepackage[american]{babel}
\usepackage{pdfsync}
\usepackage{amsmath}
\usepackage{amssymb}
\usepackage{textcomp}
\usepackage{gensymb}
\usepackage{units}
%für das Einbinden von Bildern und Tabellen
% \usepackage{here}
\usepackage{psfrag}
%für die Bibliography
\usepackage[numbers,square]{natbib}
\usepackage{hypernat}
\usepackage{ifpdf}
%Verwendung von Farben
\usepackage{xcolor}
%bedruckten Bereich der Seite anpassen
\usepackage{a4wide}
%\addtolength{\textheight}{2cm}
%\addtolength{\topmargin}{-0.7cm}
%für den Stil der Kapitel
\usepackage{titlesec}
\titleformat{\chapter}[hang]{\huge\bf}{{\huge{\thechapter \, }}}{0em}{}
%----------------------------------------------------------------------------------------------------------------------------------------------
%Stil der Seite, mit Pos. der Seitenzahl, Kapitelname, etc.... kann man auch lassen----------------------
%\usepackage{fancyhdr}
%\rhead[\fancyplain{}{\normalfont\small\sffamily\nouppercase{\leftmark}}]{\fancyplain{}{\normalfont\small\sffamily\nouppercase{\leftmark}}}%[\fancyplain{}{\bfseries\thepage}]{} %{\fancyplain{}{\bfseries\rightmark}}
%\chead[]{}%[\fancyplain{}{{\bfseries\leftmark}}]{\fancyplain{}{\bfseries\thepage}}
%\lhead[]{}
%\cfoot{\thepage}
%\fancypagestyle{plain}{%
%\fancyhf{} % clear all header and footer fields
%\fancyfoot[C]{\thepage} % except the center
%\renewcommand{\headrulewidth}{0pt}
%\renewcommand{\footrulewidth}{0pt}}
\usepackage[automark,standardstyle,markusedcase]{scrpage2}
\ihead{\headmark} \ohead{\pagemark} \cfoot{}
%
\pagestyle{empty}%für die Titelseiten zumindest, am Ende der Titelseite umschalten auf \pagestyle{fancyplain}
\pagenumbering{roman}%für die Titelseiten zumindest, am Ende der Titelseite umschalten auf \pagenumbering{arabic}
%d.h. explizit nach \tableofcontents:
%\thispagestyle{empty}
%\clearpage
%\pagestyle{fancy}%um von leere auf nummerierte Seiten umzuschalten, dies ist evtl. auch schon vorher erwünscht
%\pagenumbering{arabic}%um eine neue Zählung beginnend mit 1 zu beginnen
%------------------------je nach Kommando (pdflatex / latex) jeweilige Paketeinbindung-----------------------------
\definecolor{linkblue}{rgb}{0,0.1,0.6}
\definecolor{citegreen}{rgb}{0,0.25,0.15}%{0.1,0.5,0.4}%{0.125,0.6,0.5}
\definecolor{linkred}{rgb}{0.8,0,0.005}%{0.6,0,0.1}
\definecolor{mailviolet}{rgb}{0.3,0,0.35}%{0.6,0,0.1}
\definecolor{tumblue}{rgb}{0,0.396,0.741}
\ifpdf %pdflatex
    \usepackage[pdftex]{graphicx}
    \pdfcompresslevel=0
    \DeclareGraphicsExtensions{.jpg,.pdf,.png}%.mps}
%    \usepackage[hyperindex,pdftex,colorlinks=false]{hyperref}
%    \usepackage[hyperindex,pdfmark,pdftex,colorlinks=true,linkcolor=linkblue,citecolor=citegreen,urlcolor=linkred,filecolor=linkred]{hyperref}
    \usepackage[hyperindex,pdftex,colorlinks=true,linkcolor=tumblue,citecolor=citegreen,urlcolor=mailviolet,filecolor=linkred]{hyperref}
\else %latex && dvips
    \usepackage[dvips]{graphicx}
    \DeclareGraphicsExtensions{.eps,.ps}%.bmp,.tif,.tiff,.tga}
    %\graphicspath{{figures/}}
  \usepackage[hyperindex,pdfmark,dvips,colorlinks=true,linkcolor=tumblue,citecolor=citegreen,urlcolor=mailviolet,filecolor=linkred]{hyperref}
\fi
%für das drehen
\usepackage{rotating}
%Zeilenabstände regulieren mit \singlespacing, \onehalfspacing, \doublespacing
\usepackage{setspace}
%Schriftart Sans Serif
\renewcommand*\familydefault{\sfdefault} 
\selectlanguage{american}
